\section{Conclusion and Discussion}
This paper introduces a comprehensive modeling framework, Delphi-RF, designed to capture data revision dynamics and generate distributional revision forecasts in real-time. The application of our model extends to diverse public health data sources, encompassing outpatient COVID-19 claims data, COVID-19 antigen test data, and confirmed cases from MA-DPH.

Our framework excels in producing accurate and adaptive forecasts for the target surveillance values. These forecasts are particularly valuable for auxiliary epidemiological data streams, which are frequently used as predictive features in real-time epidemic forecasting but are often affected by the problem of data revisions. Furthermore, our framework enables timely revisions of epidemic forecasting outputs, mitigating the risk of misleading situational awareness and suboptimal decision making. Notably, Delphi-RF achieves competitive or superior forecast accuracy compared to existing methods such as NobBS and Epinowcast, while also demonstrating a 10x-100x or more improvement in computational efficiency.

Given that our method still faces challenges when epidemic disease activity levels change dramatically, particularly when encountering revision patterns not seen in historical data, a promising direction for future research is the development of a revision alerting system. This system would detect distribution shifts and predict the quality of revision forecasts based solely on early revisions, such as those available within the first one or two weeks. Such a system would complement the current framework by allowing users to proactively address potential declines in forecast accuracy and provide timely notifications to mitigate the risks associated with forecast degradation.

While our method performs robustly under typical conditions, it faces challenges during periods of dramatic shifts in epidemic activity, particularly when revision patterns diverge from those observed in historical data. A promising direction for future research is the development of a revision alerting system capable of detecting distributional shifts and providing early estimates of forecast reliability when targets are not yet available for direct evaluation. Such a system would complement the current framework by enabling users to proactively respond to potential declines in forecast accuracy and issue timely alerts to mitigate the risks associated with forecast degradation.