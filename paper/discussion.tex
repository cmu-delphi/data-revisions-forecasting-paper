\section{Conclusion and Discussion}
This paper introduces a comprehensive modeling framework, Delphi-RF, designed to capture data revision dynamics and generate distributional revision forecasts. The application of our model extends to diverse public health data sources, encompassing outpatient COVID-19 claims data, COVID-19 antigen test data, and confirmed cases from the Massachusetts Dept. of Health.

Our framework excels at generating accurate and adaptive forecasts for the target version of epidemic surveillance values. These forecasts are crucial for auxiliary epidemiological data streams, which often serve as key features in real-time epidemic forecasting. Furthermore, our framework enables timely revisions of epidemic forecasting outputs, mitigating the risk of misleading situational awareness and suboptimal decision making. In particular, Delphi-RF achieves competitive or superior forecast accuracy compared to existing methods such as nobBS and Epinowcast, while also demonstrating a 10x-100x or more improvement in computational efficiency.

Additionally, forecast accuracy can be improved by incorporating more historical data for model training, with only a moderate increase in computing time. For example, extending the training window to 365 days increases the required computation time to $63.891 \pm 0.038$ seconds for insurance claims and to $28.806 \pm 0.115$ seconds for COVID-19 cases in Massachusetts. In contrast, nobBS and Epinowcast became computationally impractical under the same extension, with runtime exceeding the 30-min cutoff per location-report-date pair before completing a single forecast. Furthermore, our model provides flexibility by allowing training on a fixed schedule while still generating daily forecasts. This flexibility allows users to adjust the training frequency based on practical needs, a feature not available in nobBS or Epinowcast.

Given that our method still faces challenges when epidemic disease activity levels change dramatically, particularly when encountering revision patterns not seen in historical data, a promising direction for future research is the development of a revision alerting system. This system would detect distribution shifts and predict the quality of revision forecasts based solely on early revisions, such as those available within the first one or two weeks. Such a system would complement the current framework by allowing users to proactively address potential declines in forecast accuracy and provide timely notifications to mitigate the risks associated with forecast degradation.


